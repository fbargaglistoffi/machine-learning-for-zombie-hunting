\documentclass[]{article}
\usepackage{lmodern}
\usepackage{amssymb,amsmath}
\usepackage{ifxetex,ifluatex}
\usepackage{fixltx2e} % provides \textsubscript
\ifnum 0\ifxetex 1\fi\ifluatex 1\fi=0 % if pdftex
  \usepackage[T1]{fontenc}
  \usepackage[utf8]{inputenc}
\else % if luatex or xelatex
  \ifxetex
    \usepackage{mathspec}
  \else
    \usepackage{fontspec}
  \fi
  \defaultfontfeatures{Ligatures=TeX,Scale=MatchLowercase}
\fi
% use upquote if available, for straight quotes in verbatim environments
\IfFileExists{upquote.sty}{\usepackage{upquote}}{}
% use microtype if available
\IfFileExists{microtype.sty}{%
\usepackage{microtype}
\UseMicrotypeSet[protrusion]{basicmath} % disable protrusion for tt fonts
}{}
\usepackage[margin=1in]{geometry}
\usepackage{hyperref}
\hypersetup{unicode=true,
            pdftitle={Why not just using a single indicator?},
            pdfauthor={Falco J. Bargagli Stoffi},
            pdfborder={0 0 0},
            breaklinks=true}
\urlstyle{same}  % don't use monospace font for urls
\usepackage{color}
\usepackage{fancyvrb}
\newcommand{\VerbBar}{|}
\newcommand{\VERB}{\Verb[commandchars=\\\{\}]}
\DefineVerbatimEnvironment{Highlighting}{Verbatim}{commandchars=\\\{\}}
% Add ',fontsize=\small' for more characters per line
\usepackage{framed}
\definecolor{shadecolor}{RGB}{248,248,248}
\newenvironment{Shaded}{\begin{snugshade}}{\end{snugshade}}
\newcommand{\AlertTok}[1]{\textcolor[rgb]{0.94,0.16,0.16}{#1}}
\newcommand{\AnnotationTok}[1]{\textcolor[rgb]{0.56,0.35,0.01}{\textbf{\textit{#1}}}}
\newcommand{\AttributeTok}[1]{\textcolor[rgb]{0.77,0.63,0.00}{#1}}
\newcommand{\BaseNTok}[1]{\textcolor[rgb]{0.00,0.00,0.81}{#1}}
\newcommand{\BuiltInTok}[1]{#1}
\newcommand{\CharTok}[1]{\textcolor[rgb]{0.31,0.60,0.02}{#1}}
\newcommand{\CommentTok}[1]{\textcolor[rgb]{0.56,0.35,0.01}{\textit{#1}}}
\newcommand{\CommentVarTok}[1]{\textcolor[rgb]{0.56,0.35,0.01}{\textbf{\textit{#1}}}}
\newcommand{\ConstantTok}[1]{\textcolor[rgb]{0.00,0.00,0.00}{#1}}
\newcommand{\ControlFlowTok}[1]{\textcolor[rgb]{0.13,0.29,0.53}{\textbf{#1}}}
\newcommand{\DataTypeTok}[1]{\textcolor[rgb]{0.13,0.29,0.53}{#1}}
\newcommand{\DecValTok}[1]{\textcolor[rgb]{0.00,0.00,0.81}{#1}}
\newcommand{\DocumentationTok}[1]{\textcolor[rgb]{0.56,0.35,0.01}{\textbf{\textit{#1}}}}
\newcommand{\ErrorTok}[1]{\textcolor[rgb]{0.64,0.00,0.00}{\textbf{#1}}}
\newcommand{\ExtensionTok}[1]{#1}
\newcommand{\FloatTok}[1]{\textcolor[rgb]{0.00,0.00,0.81}{#1}}
\newcommand{\FunctionTok}[1]{\textcolor[rgb]{0.00,0.00,0.00}{#1}}
\newcommand{\ImportTok}[1]{#1}
\newcommand{\InformationTok}[1]{\textcolor[rgb]{0.56,0.35,0.01}{\textbf{\textit{#1}}}}
\newcommand{\KeywordTok}[1]{\textcolor[rgb]{0.13,0.29,0.53}{\textbf{#1}}}
\newcommand{\NormalTok}[1]{#1}
\newcommand{\OperatorTok}[1]{\textcolor[rgb]{0.81,0.36,0.00}{\textbf{#1}}}
\newcommand{\OtherTok}[1]{\textcolor[rgb]{0.56,0.35,0.01}{#1}}
\newcommand{\PreprocessorTok}[1]{\textcolor[rgb]{0.56,0.35,0.01}{\textit{#1}}}
\newcommand{\RegionMarkerTok}[1]{#1}
\newcommand{\SpecialCharTok}[1]{\textcolor[rgb]{0.00,0.00,0.00}{#1}}
\newcommand{\SpecialStringTok}[1]{\textcolor[rgb]{0.31,0.60,0.02}{#1}}
\newcommand{\StringTok}[1]{\textcolor[rgb]{0.31,0.60,0.02}{#1}}
\newcommand{\VariableTok}[1]{\textcolor[rgb]{0.00,0.00,0.00}{#1}}
\newcommand{\VerbatimStringTok}[1]{\textcolor[rgb]{0.31,0.60,0.02}{#1}}
\newcommand{\WarningTok}[1]{\textcolor[rgb]{0.56,0.35,0.01}{\textbf{\textit{#1}}}}
\usepackage{graphicx,grffile}
\makeatletter
\def\maxwidth{\ifdim\Gin@nat@width>\linewidth\linewidth\else\Gin@nat@width\fi}
\def\maxheight{\ifdim\Gin@nat@height>\textheight\textheight\else\Gin@nat@height\fi}
\makeatother
% Scale images if necessary, so that they will not overflow the page
% margins by default, and it is still possible to overwrite the defaults
% using explicit options in \includegraphics[width, height, ...]{}
\setkeys{Gin}{width=\maxwidth,height=\maxheight,keepaspectratio}
\IfFileExists{parskip.sty}{%
\usepackage{parskip}
}{% else
\setlength{\parindent}{0pt}
\setlength{\parskip}{6pt plus 2pt minus 1pt}
}
\setlength{\emergencystretch}{3em}  % prevent overfull lines
\providecommand{\tightlist}{%
  \setlength{\itemsep}{0pt}\setlength{\parskip}{0pt}}
\setcounter{secnumdepth}{0}
% Redefines (sub)paragraphs to behave more like sections
\ifx\paragraph\undefined\else
\let\oldparagraph\paragraph
\renewcommand{\paragraph}[1]{\oldparagraph{#1}\mbox{}}
\fi
\ifx\subparagraph\undefined\else
\let\oldsubparagraph\subparagraph
\renewcommand{\subparagraph}[1]{\oldsubparagraph{#1}\mbox{}}
\fi

%%% Use protect on footnotes to avoid problems with footnotes in titles
\let\rmarkdownfootnote\footnote%
\def\footnote{\protect\rmarkdownfootnote}

%%% Change title format to be more compact
\usepackage{titling}

% Create subtitle command for use in maketitle
\providecommand{\subtitle}[1]{
  \posttitle{
    \begin{center}\large#1\end{center}
    }
}

\setlength{\droptitle}{-2em}

  \title{Why not just using a single indicator?}
    \pretitle{\vspace{\droptitle}\centering\huge}
  \posttitle{\par}
    \author{Falco J. Bargagli Stoffi}
    \preauthor{\centering\large\emph}
  \postauthor{\par}
      \predate{\centering\large\emph}
  \postdate{\par}
    \date{29/02/2020}


\begin{document}
\maketitle

\hypertarget{introduction}{%
\section{Introduction}\label{introduction}}

In this section of the Appendix we check what is the comparative
advantage of using our Machine-Learning-based definition of Persistently
Distressed Firms (PDF) rather than other commonly used deterministic
indicators (i.e., Interest Covarage Ratio {[}ICR{]}, negative added
value, etc.).

\par

The first and more obvious advantage is that the PDF indicator can be
constructed even in the presence of missing data. As we show in the
Section of the paper about missing data, there are significant
missingness patterns in the two variables that compose the ICR indicator
(EBIT and Interest Paid).

\par

The second advantage is that, while ICR and the other indicators are
based on a deterministic definition of zombies, PDF is based on a
probabilistic definition that can be tuned to embody different
``likelihoods'' of being zombie.

\par

The third advantage is that our PDF indicator can be constructed using
the information coming from multiple indicators and not just using the
information on a single one.

\par

Beside these clear advantages, there are more subtle benefits that can
be obtained by using PDF instead of ICR. To highlight these advantages
we focus on a particular dimension of our PDF definition. To be a PDF a
firm needs to be at high risk of failure (but not failing) for three
consecutive years. Our PDF definition focuses on those firms that ``are
predicted to fail, but are surviving'' (following the ML literature we
could call them \textit{false positives}). Hence, a good indicator of
``\textit{zombieness}'' should be able to discriminate between
\textit{false positives} and \textit{true positives}.

\par

In this spirit, we develop a two stage algorithm to highlight which are
the best predictors of \textit{false positives}. In the first stage, we
train a Logit model to predict the probability of failure:
\begin{equation}
 f_{LOGIT}(x) = \hat{p}_i(Y_i = 1 | X_i = x),
\end{equation} and to obtain the fitted values \(\hat{y}_i\). In the
second stage, we fit a LASSO just on those observations with
\(\hat{y}_i=1\) (\textit{positives}) to get the most important
predictors (namely, those with non-zero coefficient) for the
\textit{false positives}.

\par

In the following code chuncks the implementation of this ``two-stage''
algorithm.

\hypertarget{load-packages-and-data}{%
\subsection{Load Packages and Data}\label{load-packages-and-data}}

In this chunk we load the packages and the data used for the analysis,
and we split the overall sample in a \textit{training} and a
\textit{test} set.

\begin{Shaded}
\begin{Highlighting}[]
\KeywordTok{options}\NormalTok{(}\DataTypeTok{java.parameters =} \StringTok{"-Xmx50g"}\NormalTok{)}
\KeywordTok{library}\NormalTok{(rJava)}
\KeywordTok{library}\NormalTok{(bartMachine)}
\KeywordTok{library}\NormalTok{(haven)}
\KeywordTok{library}\NormalTok{(plyr)}
\KeywordTok{library}\NormalTok{(dplyr)}
\KeywordTok{library}\NormalTok{(PRROC)}
\KeywordTok{library}\NormalTok{(caret)}
\KeywordTok{library}\NormalTok{(glmnet)}

\CommentTok{# Load Data}
\NormalTok{data <-}\StringTok{ }\KeywordTok{read_dta}\NormalTok{(}\StringTok{"analysis_data_indicators.dta"}\NormalTok{)}

\CommentTok{# Initialize Data}
\KeywordTok{names}\NormalTok{(data)[}\KeywordTok{names}\NormalTok{(data) }\OperatorTok{==}\StringTok{ 'GUO___BvD_ID_number'}\NormalTok{] <-}\StringTok{ 'guo'}
\NormalTok{data}\OperatorTok{$}\NormalTok{control <-}\StringTok{ }\KeywordTok{ifelse}\NormalTok{(data}\OperatorTok{$}\NormalTok{guo}\OperatorTok{==}\StringTok{""}\NormalTok{, }\DecValTok{0}\NormalTok{, }\DecValTok{1}\NormalTok{)}
\NormalTok{data}\OperatorTok{$}\NormalTok{nace <-}\StringTok{ }\KeywordTok{as.factor}\NormalTok{(data}\OperatorTok{$}\NormalTok{nace)}
\NormalTok{data}\OperatorTok{$}\NormalTok{area <-}\StringTok{ }\KeywordTok{as.factor}\NormalTok{(data}\OperatorTok{$}\NormalTok{area)}
\KeywordTok{levels}\NormalTok{(data}\OperatorTok{$}\NormalTok{nace) <-}\StringTok{ }\KeywordTok{floor}\NormalTok{(}\KeywordTok{as.numeric}\NormalTok{(}\KeywordTok{levels}\NormalTok{(data}\OperatorTok{$}\NormalTok{nace))}\OperatorTok{/}\DecValTok{100}\NormalTok{) }

\CommentTok{#OMIITED DATA}
\NormalTok{lagged_variables <-}\StringTok{ }\KeywordTok{c}\NormalTok{(}\StringTok{"failure"}\NormalTok{, }\StringTok{"iso"}\NormalTok{, }\StringTok{"control"}\NormalTok{, }\StringTok{"nace"}\NormalTok{,}
                      \StringTok{"shareholders_funds"}\NormalTok{, }\StringTok{"added_value"}\NormalTok{,}
                      \StringTok{"cash_flow"}\NormalTok{, }\StringTok{"ebitda"}\NormalTok{, }\StringTok{"fin_rev"}\NormalTok{,}
                      \StringTok{"liquidity_ratio"}\NormalTok{, }\StringTok{"total_assets"}\NormalTok{,}
                      \StringTok{"depr"}\NormalTok{, }\StringTok{"long_term_debt"}\NormalTok{, }\StringTok{"employees"}\NormalTok{,}
                      \StringTok{"materials"}\NormalTok{, }\StringTok{"loans"}\NormalTok{, }\StringTok{"wage_bill"}\NormalTok{,}
                      \StringTok{"tfp_acf"}\NormalTok{, }\StringTok{"fixed_assets"}\NormalTok{, }\StringTok{"tax"}\NormalTok{,}
                      \StringTok{"current_liabilities"}\NormalTok{, }\StringTok{"current_assets"}\NormalTok{,}
                      \StringTok{"fin_expenses"}\NormalTok{, }\StringTok{"int_paid"}\NormalTok{,}
                      \StringTok{"solvency_ratio"}\NormalTok{, }\StringTok{"net_income"}\NormalTok{,}
                      \StringTok{"revenue"}\NormalTok{, }\StringTok{"consdummy"}\NormalTok{, }\StringTok{"capital_intensity"}\NormalTok{,}
                      \StringTok{"fin_cons100"}\NormalTok{, }\StringTok{"inv"}\NormalTok{, }\StringTok{"ICR_failure"}\NormalTok{,}
                      \StringTok{"interest_diff"}\NormalTok{, }\StringTok{"NEG_VA"}\NormalTok{, }\StringTok{"real_SA"}\NormalTok{,}
                      \StringTok{"Z_score"}\NormalTok{, }\StringTok{"misallocated_fixed"}\NormalTok{,}
                      \StringTok{"profitability"}\NormalTok{, }\StringTok{"area"}\NormalTok{, }\StringTok{"dummy_patents"}\NormalTok{,}
                      \StringTok{"dummy_trademark"}\NormalTok{,}\StringTok{"financial_sustainability"}\NormalTok{,}
                      \StringTok{"liquidity_return"}\NormalTok{, }\StringTok{"int_fixed_assets"}\NormalTok{)}
\NormalTok{omitted <-}\StringTok{ }\KeywordTok{na.omit}\NormalTok{(data[lagged_variables])}
\NormalTok{predictors <-}\StringTok{ }\KeywordTok{c}\NormalTok{(}\StringTok{"control"}\NormalTok{, }\StringTok{"nace"}\NormalTok{, }\StringTok{"shareholders_funds"}\NormalTok{,}
                \StringTok{"added_value"}\NormalTok{, }\StringTok{"cash_flow"}\NormalTok{, }\StringTok{"ebitda"}\NormalTok{,}
                \StringTok{"fin_rev"}\NormalTok{, }\StringTok{"liquidity_ratio"}\NormalTok{, }\StringTok{"total_assets"}\NormalTok{,}
                \StringTok{"depr"}\NormalTok{, }\StringTok{"long_term_debt"}\NormalTok{, }\StringTok{"employees"}\NormalTok{,}
                \StringTok{"materials"}\NormalTok{, }\StringTok{"loans"}\NormalTok{, }\StringTok{"wage_bill"}\NormalTok{, }\StringTok{"tfp_acf"}\NormalTok{,}
                \StringTok{"fixed_assets"}\NormalTok{, }\StringTok{"tax"}\NormalTok{, }\StringTok{"current_liabilities"}\NormalTok{,}
                \StringTok{"current_assets"}\NormalTok{, }\StringTok{"fin_expenses"}\NormalTok{, }\StringTok{"int_paid"}\NormalTok{,}
                \StringTok{"solvency_ratio"}\NormalTok{, }\StringTok{"net_income"}\NormalTok{, }\StringTok{"revenue"}\NormalTok{,}
                \StringTok{"consdummy"}\NormalTok{, }\StringTok{"capital_intensity"}\NormalTok{, }\StringTok{"fin_cons100"}\NormalTok{,}
                \StringTok{"inv"}\NormalTok{, }\StringTok{"ICR_failure"}\NormalTok{, }\StringTok{"interest_diff"}\NormalTok{, }\StringTok{"NEG_VA"}\NormalTok{,}
                \StringTok{"real_SA"}\NormalTok{, }\StringTok{"misallocated_fixed"}\NormalTok{, }\StringTok{"profitability"}\NormalTok{,}
                \StringTok{"area"}\NormalTok{, }\StringTok{"dummy_patents"}\NormalTok{, }\StringTok{"dummy_trademark"}\NormalTok{,}
                \StringTok{"financial_sustainability"}\NormalTok{, }\StringTok{"liquidity_return"}\NormalTok{,}
                \StringTok{"int_fixed_assets"}\NormalTok{)}
\NormalTok{formula <-}\StringTok{ }\KeywordTok{as.formula}\NormalTok{(}\KeywordTok{paste}\NormalTok{(}\StringTok{"as.factor(failure) ~"}\NormalTok{,}
                            \KeywordTok{paste}\NormalTok{(predictors, }\DataTypeTok{collapse=}\StringTok{"+"}\NormalTok{)))}

\NormalTok{### Define samples}
\KeywordTok{set.seed}\NormalTok{(}\DecValTok{123}\NormalTok{)}
\NormalTok{train_sample <-}\StringTok{ }\KeywordTok{sample}\NormalTok{(}\KeywordTok{seq_len}\NormalTok{(}\KeywordTok{nrow}\NormalTok{(omitted)), }\DataTypeTok{size =} \KeywordTok{nrow}\NormalTok{(omitted)}\OperatorTok{*}\FloatTok{0.5}\NormalTok{) }
\NormalTok{train <-}\StringTok{ }\KeywordTok{as.data.frame}\NormalTok{(omitted[train_sample,])}
\NormalTok{test <-}\StringTok{ }\KeywordTok{as.data.frame}\NormalTok{(omitted[}\OperatorTok{-}\NormalTok{train_sample,])}
\end{Highlighting}
\end{Shaded}

In this chunck we contruct the LOGIT model, we get the predicted
probabilities and the confusion matrix. We use as a threshold 0.3
(namely, \(\hat{p}_i(Y_i = 1 | X_i = x)>0.3 \rightarrow \hat{y}_i=1\)),
however this threshold can be moved up to 0.5 without any meaninful
change.

\begin{Shaded}
\begin{Highlighting}[]
\NormalTok{log <-}\StringTok{ }\KeywordTok{glm}\NormalTok{(formula, }\DataTypeTok{family =}\NormalTok{ binomial, }\DataTypeTok{data =}\NormalTok{ train)}
\NormalTok{prob_pred <-}\StringTok{ }\KeywordTok{predict}\NormalTok{(log, }\DataTypeTok{type =} \StringTok{'response'}\NormalTok{, }\DataTypeTok{newdata =}\NormalTok{ test)}
\NormalTok{prediction <-}\StringTok{ }\KeywordTok{as.numeric}\NormalTok{(prob_pred }\OperatorTok{>}\StringTok{ }\FloatTok{0.3}\NormalTok{) }\CommentTok{# change up to 0.5}
\NormalTok{cmlog=}\KeywordTok{table}\NormalTok{(test}\OperatorTok{$}\NormalTok{failure,prediction)}
\NormalTok{cmlog}
\end{Highlighting}
\end{Shaded}

\begin{verbatim}
##    prediction
##         0     1
##   0 43951   605
##   1  1381   473
\end{verbatim}

In this chunck we use a LASSO model to select the variables with the
highest predictive power for \textit{false positives}.

\begin{Shaded}
\begin{Highlighting}[]
\NormalTok{test}\OperatorTok{$}\NormalTok{prediction <-}\StringTok{ }\NormalTok{prediction}
\NormalTok{positive <-}\StringTok{ }\NormalTok{test[}\KeywordTok{which}\NormalTok{(test}\OperatorTok{$}\NormalTok{prediction}\OperatorTok{==}\DecValTok{1}\NormalTok{),]}
\NormalTok{positive}\OperatorTok{$}\NormalTok{iso <-}\StringTok{ }\KeywordTok{as.numeric}\NormalTok{(}\KeywordTok{as.factor}\NormalTok{(positive}\OperatorTok{$}\NormalTok{iso))}
\NormalTok{positive}\OperatorTok{$}\NormalTok{control <-}\StringTok{ }\KeywordTok{as.numeric}\NormalTok{(}\KeywordTok{as.factor}\NormalTok{(positive}\OperatorTok{$}\NormalTok{control))}
\NormalTok{x <-}\StringTok{ }\KeywordTok{as.matrix}\NormalTok{(}\KeywordTok{as.data.frame}\NormalTok{(}\KeywordTok{lapply}\NormalTok{(positive[predictors], as.numeric)))}
\NormalTok{y <-}\StringTok{ }\KeywordTok{as.numeric}\NormalTok{(positive}\OperatorTok{$}\NormalTok{prediction}\OperatorTok{==}\DecValTok{1} \OperatorTok{&}\StringTok{ }\NormalTok{positive}\OperatorTok{$}\NormalTok{failure}\OperatorTok{==}\DecValTok{0}\NormalTok{)}

\NormalTok{mod <-}\StringTok{ }\KeywordTok{cv.glmnet}\NormalTok{(x , y, }\DataTypeTok{alpha=}\DecValTok{1}\NormalTok{)}
\KeywordTok{as.matrix}\NormalTok{(}\KeywordTok{coef}\NormalTok{(mod, mod}\OperatorTok{$}\NormalTok{lambda}\FloatTok{.1}\NormalTok{se))}
\end{Highlighting}
\end{Shaded}

\begin{verbatim}
##                                     1
## (Intercept)               0.495101483
## control                   0.084239097
## nace                      0.000000000
## shareholders_funds        0.000000000
## added_value               0.000000000
## cash_flow                 0.000000000
## ebitda                    0.000000000
## fin_rev                   0.000000000
## liquidity_ratio           0.000000000
## total_assets              0.000000000
## depr                      0.000000000
## long_term_debt            0.000000000
## employees                 0.000000000
## materials                 0.000000000
## loans                     0.000000000
## wage_bill                 0.000000000
## tfp_acf                   0.000000000
## fixed_assets              0.000000000
## tax                       0.000000000
## current_liabilities       0.000000000
## current_assets            0.000000000
## fin_expenses              0.000000000
## int_paid                  0.000000000
## solvency_ratio            0.001577534
## net_income                0.000000000
## revenue                   0.000000000
## consdummy                 0.000000000
## capital_intensity         0.000000000
## fin_cons100               0.000000000
## inv                       0.000000000
## ICR_failure               0.000000000
## interest_diff             0.052601540
## NEG_VA                   -0.045676512
## real_SA                   0.000000000
## misallocated_fixed        0.000000000
## profitability            -0.005923519
## area                      0.005088934
## dummy_patents             0.000000000
## dummy_trademark           0.000000000
## financial_sustainability  0.000000000
## liquidity_return          0.436828513
## int_fixed_assets          0.000000000
\end{verbatim}

\begin{Shaded}
\begin{Highlighting}[]
\KeywordTok{row.names}\NormalTok{(}\KeywordTok{as.matrix}\NormalTok{(}\KeywordTok{coef}\NormalTok{(mod, mod}\OperatorTok{$}\NormalTok{lambda}\FloatTok{.1}\NormalTok{se)))[}\KeywordTok{which}\NormalTok{(}\KeywordTok{as.matrix}\NormalTok{(}\KeywordTok{coef}\NormalTok{(mod, mod}\OperatorTok{$}\NormalTok{lambda}\FloatTok{.1}\NormalTok{se))}\OperatorTok{!=}\DecValTok{0}\NormalTok{)]}
\end{Highlighting}
\end{Shaded}

\begin{verbatim}
## [1] "(Intercept)"      "control"          "solvency_ratio"  
## [4] "interest_diff"    "NEG_VA"           "profitability"   
## [7] "area"             "liquidity_return"
\end{verbatim}

The indicator that seems to have the best disciminative power are:
\textit{control, solvency ratio, BID, negative AV, profitability, area, liquidity}.
Hence, these indicators are the best to capture the component of
``resistence'' among highly distressed firms that makes them zombies.


\end{document}

\documentclass[]{article}
\usepackage{lmodern}
\usepackage{amssymb,amsmath}
\usepackage{ifxetex,ifluatex}
\usepackage{fixltx2e} % provides \textsubscript
\ifnum 0\ifxetex 1\fi\ifluatex 1\fi=0 % if pdftex
  \usepackage[T1]{fontenc}
  \usepackage[utf8]{inputenc}
\else % if luatex or xelatex
  \ifxetex
    \usepackage{mathspec}
  \else
    \usepackage{fontspec}
  \fi
  \defaultfontfeatures{Ligatures=TeX,Scale=MatchLowercase}
\fi
% use upquote if available, for straight quotes in verbatim environments
\IfFileExists{upquote.sty}{\usepackage{upquote}}{}
% use microtype if available
\IfFileExists{microtype.sty}{%
\usepackage{microtype}
\UseMicrotypeSet[protrusion]{basicmath} % disable protrusion for tt fonts
}{}
\usepackage[margin=1in]{geometry}
\usepackage{hyperref}
\hypersetup{unicode=true,
            pdftitle={PDF v. ICR},
            pdfauthor={Falco J. Bargagli Stoffi},
            pdfborder={0 0 0},
            breaklinks=true}
\urlstyle{same}  % don't use monospace font for urls
\usepackage{color}
\usepackage{fancyvrb}
\newcommand{\VerbBar}{|}
\newcommand{\VERB}{\Verb[commandchars=\\\{\}]}
\DefineVerbatimEnvironment{Highlighting}{Verbatim}{commandchars=\\\{\}}
% Add ',fontsize=\small' for more characters per line
\usepackage{framed}
\definecolor{shadecolor}{RGB}{248,248,248}
\newenvironment{Shaded}{\begin{snugshade}}{\end{snugshade}}
\newcommand{\AlertTok}[1]{\textcolor[rgb]{0.94,0.16,0.16}{#1}}
\newcommand{\AnnotationTok}[1]{\textcolor[rgb]{0.56,0.35,0.01}{\textbf{\textit{#1}}}}
\newcommand{\AttributeTok}[1]{\textcolor[rgb]{0.77,0.63,0.00}{#1}}
\newcommand{\BaseNTok}[1]{\textcolor[rgb]{0.00,0.00,0.81}{#1}}
\newcommand{\BuiltInTok}[1]{#1}
\newcommand{\CharTok}[1]{\textcolor[rgb]{0.31,0.60,0.02}{#1}}
\newcommand{\CommentTok}[1]{\textcolor[rgb]{0.56,0.35,0.01}{\textit{#1}}}
\newcommand{\CommentVarTok}[1]{\textcolor[rgb]{0.56,0.35,0.01}{\textbf{\textit{#1}}}}
\newcommand{\ConstantTok}[1]{\textcolor[rgb]{0.00,0.00,0.00}{#1}}
\newcommand{\ControlFlowTok}[1]{\textcolor[rgb]{0.13,0.29,0.53}{\textbf{#1}}}
\newcommand{\DataTypeTok}[1]{\textcolor[rgb]{0.13,0.29,0.53}{#1}}
\newcommand{\DecValTok}[1]{\textcolor[rgb]{0.00,0.00,0.81}{#1}}
\newcommand{\DocumentationTok}[1]{\textcolor[rgb]{0.56,0.35,0.01}{\textbf{\textit{#1}}}}
\newcommand{\ErrorTok}[1]{\textcolor[rgb]{0.64,0.00,0.00}{\textbf{#1}}}
\newcommand{\ExtensionTok}[1]{#1}
\newcommand{\FloatTok}[1]{\textcolor[rgb]{0.00,0.00,0.81}{#1}}
\newcommand{\FunctionTok}[1]{\textcolor[rgb]{0.00,0.00,0.00}{#1}}
\newcommand{\ImportTok}[1]{#1}
\newcommand{\InformationTok}[1]{\textcolor[rgb]{0.56,0.35,0.01}{\textbf{\textit{#1}}}}
\newcommand{\KeywordTok}[1]{\textcolor[rgb]{0.13,0.29,0.53}{\textbf{#1}}}
\newcommand{\NormalTok}[1]{#1}
\newcommand{\OperatorTok}[1]{\textcolor[rgb]{0.81,0.36,0.00}{\textbf{#1}}}
\newcommand{\OtherTok}[1]{\textcolor[rgb]{0.56,0.35,0.01}{#1}}
\newcommand{\PreprocessorTok}[1]{\textcolor[rgb]{0.56,0.35,0.01}{\textit{#1}}}
\newcommand{\RegionMarkerTok}[1]{#1}
\newcommand{\SpecialCharTok}[1]{\textcolor[rgb]{0.00,0.00,0.00}{#1}}
\newcommand{\SpecialStringTok}[1]{\textcolor[rgb]{0.31,0.60,0.02}{#1}}
\newcommand{\StringTok}[1]{\textcolor[rgb]{0.31,0.60,0.02}{#1}}
\newcommand{\VariableTok}[1]{\textcolor[rgb]{0.00,0.00,0.00}{#1}}
\newcommand{\VerbatimStringTok}[1]{\textcolor[rgb]{0.31,0.60,0.02}{#1}}
\newcommand{\WarningTok}[1]{\textcolor[rgb]{0.56,0.35,0.01}{\textbf{\textit{#1}}}}
\usepackage{graphicx,grffile}
\makeatletter
\def\maxwidth{\ifdim\Gin@nat@width>\linewidth\linewidth\else\Gin@nat@width\fi}
\def\maxheight{\ifdim\Gin@nat@height>\textheight\textheight\else\Gin@nat@height\fi}
\makeatother
% Scale images if necessary, so that they will not overflow the page
% margins by default, and it is still possible to overwrite the defaults
% using explicit options in \includegraphics[width, height, ...]{}
\setkeys{Gin}{width=\maxwidth,height=\maxheight,keepaspectratio}
\IfFileExists{parskip.sty}{%
\usepackage{parskip}
}{% else
\setlength{\parindent}{0pt}
\setlength{\parskip}{6pt plus 2pt minus 1pt}
}
\setlength{\emergencystretch}{3em}  % prevent overfull lines
\providecommand{\tightlist}{%
  \setlength{\itemsep}{0pt}\setlength{\parskip}{0pt}}
\setcounter{secnumdepth}{0}
% Redefines (sub)paragraphs to behave more like sections
\ifx\paragraph\undefined\else
\let\oldparagraph\paragraph
\renewcommand{\paragraph}[1]{\oldparagraph{#1}\mbox{}}
\fi
\ifx\subparagraph\undefined\else
\let\oldsubparagraph\subparagraph
\renewcommand{\subparagraph}[1]{\oldsubparagraph{#1}\mbox{}}
\fi

%%% Use protect on footnotes to avoid problems with footnotes in titles
\let\rmarkdownfootnote\footnote%
\def\footnote{\protect\rmarkdownfootnote}

%%% Change title format to be more compact
\usepackage{titling}

% Create subtitle command for use in maketitle
\newcommand{\subtitle}[1]{
  \posttitle{
    \begin{center}\large#1\end{center}
    }
}

\setlength{\droptitle}{-2em}

  \title{PDF v. ICR}
    \pretitle{\vspace{\droptitle}\centering\huge}
  \posttitle{\par}
    \author{Falco J. Bargagli Stoffi}
    \preauthor{\centering\large\emph}
  \postauthor{\par}
      \predate{\centering\large\emph}
  \postdate{\par}
    \date{2/12/2018}


\begin{document}
\maketitle

\hypertarget{why-not-just-using-icr}{%
\section{Why not just using ICR?}\label{why-not-just-using-icr}}

In this section of the Appendix we check what is the comparative
advantage of using our Machine-Learning-based definition of Persistently
Distressed Firms (PDF) rather than other commonly used deterministic
indicators (i.e., Interest Covarage Ratio {[}ICR{]}, negative added
value).

\par

The first and more obvious advantage is that the PDF indicator can be
constructed even in the presence of missing data. As we saw in the
Section of the paper about missing data, there are significant
missingness patterns in the two variables that compose the ICR (EBIT and
Interest Paid).

\par

The second advantage is that, while ICR entails a deterministic
definition of zombies, PDF represents a probabilistic based definition
that can be tuned to embody different ``likelihoods'' of being zombie.

\par

Beside these clear advantages, there are more subtle benefits that can
be obtained by using PDF instead of ICR. To highlight these advantages
we focus on a particular dimension of our PDF definition. To be a PDF a
firm needs to be at high risk of failure (but not failing) for three
consecutive years. Our PDF definition focuses on those firms that ``are
predicted to fail, but are surviving'' (following the ML literature we
could call them \textit{false positives}). Hence, a good indicator of
``\textit{zombieness}'' should be able to discriminate between
\textit{false positives} and \textit{true positives}.

\par

In this spirit, we develop a two stage algorithm to highlight which are
the best predictors of \textit{false positives}. In the first stage, we
train a Logit model to predict the probability of failure:
\begin{equation}
 f_{LOGIT}(x) = \hat{p}_i(Y_i = 1 | X_i = x),
\end{equation} and to obtain the fitted values \(\hat{y}_i\). In the
second stage, we fit a LASSO just on those observations with
\(\hat{y}_i=1\) (\textit{positives}) to get the most important
predictors (namely, those with non-zero coefficient) for the
\textit{false positives}.

\par

In the following code chuncks the implementation of this ``two-stage''
algorithm.

\hypertarget{load-packages-and-data}{%
\subsection{Load Packages and Data}\label{load-packages-and-data}}

In this chunk we load the packages and the data used for the analysis,
and we split the overall sample in a \textit{training} and a
\textit{test} set.

\begin{Shaded}
\begin{Highlighting}[]
\KeywordTok{options}\NormalTok{(}\DataTypeTok{java.parameters =} \StringTok{"-Xmx50g"}\NormalTok{)}
\KeywordTok{library}\NormalTok{(rJava)}
\KeywordTok{library}\NormalTok{(bartMachine)}
\KeywordTok{library}\NormalTok{(haven)}
\KeywordTok{library}\NormalTok{(plyr)}
\KeywordTok{library}\NormalTok{(dplyr)}
\KeywordTok{library}\NormalTok{(PRROC)}
\KeywordTok{library}\NormalTok{(caret)}
\KeywordTok{library}\NormalTok{(glmnet)}

\CommentTok{#Upload dati}
\KeywordTok{setwd}\NormalTok{(}\StringTok{"G:}\CharTok{\textbackslash{}\textbackslash{}}\StringTok{Il mio Drive}\CharTok{\textbackslash{}\textbackslash{}}\StringTok{Research}\CharTok{\textbackslash{}\textbackslash{}}\StringTok{Italian Firms}\CharTok{\textbackslash{}\textbackslash{}}\StringTok{Zombie Hunting New Data"}\NormalTok{)}

\CommentTok{#LOAD DATA}

\NormalTok{data <-}\StringTok{ }\KeywordTok{read_dta}\NormalTok{(}\StringTok{"data_lagged_10_07.dta"}\NormalTok{)}

\CommentTok{#OMIITED DATA}
\NormalTok{myvariables <-}\StringTok{ }\KeywordTok{c}\NormalTok{(}\StringTok{"iso"}\NormalTok{, }\StringTok{"tfp_acf"}\NormalTok{,  }\StringTok{"Number_of_patents"}\NormalTok{, }\StringTok{"Number_of_trademarks"}\NormalTok{,}
                 \StringTok{"consdummy"}\NormalTok{, }\StringTok{"control"}\NormalTok{, }\StringTok{"failure"}\NormalTok{, }\StringTok{"nace_2"}\NormalTok{,}\StringTok{"fin_rev"}\NormalTok{, }\StringTok{"int_paid"}\NormalTok{,}
                 \StringTok{"ebitda"}\NormalTok{, }\StringTok{"cash_flow"}\NormalTok{, }\StringTok{"depr"}\NormalTok{, }\StringTok{"revenue"}\NormalTok{, }\StringTok{"total_assets"}\NormalTok{, }
                 \StringTok{"long_term_debt"}\NormalTok{, }\StringTok{"employees"}\NormalTok{, }\StringTok{"added_value"}\NormalTok{, }\StringTok{"materials"}\NormalTok{, }\StringTok{"wage_bill"}\NormalTok{,}
                 \StringTok{"loans"}\NormalTok{ , }\StringTok{"int_fixed_assets"}\NormalTok{, }\StringTok{"fixed_assets"}\NormalTok{, }\StringTok{"current_liabilities"}\NormalTok{,}
                 \StringTok{"liquidity_ratio"}\NormalTok{,  }\StringTok{"solvency_ratio"}\NormalTok{, }\StringTok{"current_assets"}\NormalTok{, }\StringTok{"fin_expenses"}\NormalTok{,}
                 \StringTok{"net_income"}\NormalTok{, }\StringTok{"fin_cons"}\NormalTok{, }\StringTok{"fin_cons100"}\NormalTok{, }\StringTok{"inv"}\NormalTok{,  }\StringTok{"ICR_t"}\NormalTok{, }\StringTok{"time"}\NormalTok{,}
                 \StringTok{"real_SA"}\NormalTok{, }\StringTok{"shareholders_funds"}\NormalTok{, }\StringTok{"NEG_VA"}\NormalTok{, }\StringTok{"ICR_failure"}\NormalTok{, }\StringTok{"profitability"}\NormalTok{,}
                 \StringTok{"misallocated_fixed"}\NormalTok{,}\StringTok{"interest_diff"}\NormalTok{)}
\CommentTok{#capital_intensity, labour_product,retained_earnings, firm_value excluded: highly missing}
\NormalTok{dati <-}\StringTok{ }\KeywordTok{na.omit}\NormalTok{(data[myvariables])}
\NormalTok{predictors <-}\StringTok{ }\KeywordTok{c}\NormalTok{(}\StringTok{"iso"}\NormalTok{, }\StringTok{"tfp_acf"}\NormalTok{,  }\StringTok{"Number_of_patents"}\NormalTok{, }\StringTok{"Number_of_trademarks"}\NormalTok{,}
                \StringTok{"consdummy"}\NormalTok{, }\StringTok{"control"}\NormalTok{, }\StringTok{"nace_2"}\NormalTok{,}\StringTok{"fin_rev"}\NormalTok{, }\StringTok{"int_paid"}\NormalTok{, }\StringTok{"ebitda"}\NormalTok{,}
                \StringTok{"cash_flow"}\NormalTok{, }\StringTok{"depr"}\NormalTok{, }\StringTok{"revenue"}\NormalTok{, }\StringTok{"total_assets"}\NormalTok{, }\StringTok{"long_term_debt"}\NormalTok{,}
                \StringTok{"employees"}\NormalTok{, }\StringTok{"added_value"}\NormalTok{, }\StringTok{"materials"}\NormalTok{, }\StringTok{"wage_bill"}\NormalTok{, }\StringTok{"loans"}\NormalTok{ ,}
                \StringTok{"int_fixed_assets"}\NormalTok{,  }\StringTok{"fixed_assets"}\NormalTok{, }\StringTok{"current_liabilities"}\NormalTok{,}
                \StringTok{"liquidity_ratio"}\NormalTok{,  }\StringTok{"solvency_ratio"}\NormalTok{, }\StringTok{"current_assets"}\NormalTok{,}
                 \StringTok{"fin_expenses"}\NormalTok{, }\StringTok{"net_income"}\NormalTok{, }\StringTok{"fin_cons"}\NormalTok{, }\StringTok{"fin_cons100"}\NormalTok{, }\StringTok{"inv"}\NormalTok{,}
                \StringTok{"ICR_t"}\NormalTok{, }\StringTok{"time"}\NormalTok{, }\StringTok{"real_SA"}\NormalTok{, }\StringTok{"shareholders_funds"}\NormalTok{, }\StringTok{"NEG_VA"}\NormalTok{,}
                \StringTok{"ICR_failure"}\NormalTok{, }\StringTok{"profitability"}\NormalTok{, }\StringTok{"misallocated_fixed"}\NormalTok{, }
                \StringTok{"interest_diff"}\NormalTok{)}

\NormalTok{### Define samples}
\KeywordTok{set.seed}\NormalTok{(}\DecValTok{123}\NormalTok{)}
\NormalTok{train_sample <-}\StringTok{ }\KeywordTok{sample}\NormalTok{(}\KeywordTok{seq_len}\NormalTok{(}\KeywordTok{nrow}\NormalTok{(dati)), }\DataTypeTok{size =} \KeywordTok{nrow}\NormalTok{(dati)}\OperatorTok{*}\FloatTok{0.5}\NormalTok{) }
\NormalTok{train <-}\StringTok{ }\KeywordTok{as.data.frame}\NormalTok{(dati[train_sample,])}
\NormalTok{test <-}\StringTok{ }\KeywordTok{as.data.frame}\NormalTok{(dati[}\OperatorTok{-}\NormalTok{train_sample,])}
\end{Highlighting}
\end{Shaded}

In this chunck we contruct the LOGIT model, we get the predicted
probabilities and the confusion matrix. We use as a threshold 0.3
(namely, \(\hat{p}_i(Y_i = 1 | X_i = x)>0.3 \rightarrow \hat{y}_i=1\)),
however this threshold can be moved up to 0.5 without any meaninful
change.

\begin{Shaded}
\begin{Highlighting}[]
\NormalTok{log =}\StringTok{ }\KeywordTok{glm}\NormalTok{(}\DataTypeTok{formula =}\NormalTok{ train}\OperatorTok{$}\NormalTok{failure }\OperatorTok{~}\StringTok{ }\NormalTok{., }\DataTypeTok{family =}\NormalTok{ binomial, }\DataTypeTok{data =}\NormalTok{ train)}
\NormalTok{prob_pred =}\StringTok{ }\KeywordTok{predict}\NormalTok{(log, }\DataTypeTok{type =} \StringTok{'response'}\NormalTok{, }\DataTypeTok{newdata =}\NormalTok{ test)}
\NormalTok{prediction <-}\StringTok{ }\KeywordTok{as.numeric}\NormalTok{(prob_pred }\OperatorTok{>}\StringTok{ }\FloatTok{0.3}\NormalTok{) }\CommentTok{# change up to 0.5}
\NormalTok{cmlog=}\KeywordTok{table}\NormalTok{(test}\OperatorTok{$}\NormalTok{failure,prediction)}
\NormalTok{cmlog}
\end{Highlighting}
\end{Shaded}

\begin{verbatim}
##    prediction
##         0     1
##   0 63392   326
##   1  1830   301
\end{verbatim}

In this chunck we use a LASSO model to select the variables with the
highest predictive power for \textit{false positives}.

\begin{Shaded}
\begin{Highlighting}[]
\NormalTok{test}\OperatorTok{$}\NormalTok{prediction <-}\StringTok{ }\NormalTok{prediction}
\NormalTok{positive <-}\StringTok{ }\NormalTok{test[}\KeywordTok{which}\NormalTok{(test}\OperatorTok{$}\NormalTok{prediction}\OperatorTok{==}\DecValTok{1}\NormalTok{),]}
\NormalTok{positive}\OperatorTok{$}\NormalTok{iso <-}\StringTok{ }\KeywordTok{as.numeric}\NormalTok{(}\KeywordTok{as.factor}\NormalTok{(positive}\OperatorTok{$}\NormalTok{iso))}
\NormalTok{positive}\OperatorTok{$}\NormalTok{control <-}\StringTok{ }\KeywordTok{as.numeric}\NormalTok{(}\KeywordTok{as.factor}\NormalTok{(positive}\OperatorTok{$}\NormalTok{control))}
\NormalTok{x<-}\KeywordTok{as.matrix}\NormalTok{(positive[, }\KeywordTok{c}\NormalTok{(}\StringTok{"iso"}\NormalTok{, }\StringTok{"tfp_acf"}\NormalTok{,  }\StringTok{"Number_of_patents"}\NormalTok{, }\StringTok{"Number_of_trademarks"}\NormalTok{,}
                          \StringTok{"consdummy"}\NormalTok{, }\StringTok{"control"}\NormalTok{,}\StringTok{"nace_2"}\NormalTok{,}\StringTok{"fin_rev"}\NormalTok{, }\StringTok{"int_paid"}\NormalTok{, }\StringTok{"ebitda"}\NormalTok{,}
                          \StringTok{"cash_flow"}\NormalTok{, }\StringTok{"depr"}\NormalTok{, }\StringTok{"revenue"}\NormalTok{, }\StringTok{"total_assets"}\NormalTok{, }\StringTok{"long_term_debt"}\NormalTok{,}
                          \StringTok{"employees"}\NormalTok{, }\StringTok{"added_value"}\NormalTok{, }\StringTok{"materials"}\NormalTok{, }\StringTok{"wage_bill"}\NormalTok{, }\StringTok{"loans"}\NormalTok{ ,}
                          \StringTok{"int_fixed_assets"}\NormalTok{,}\StringTok{"fixed_assets"}\NormalTok{, }\StringTok{"current_liabilities"}\NormalTok{,}
                          \StringTok{"liquidity_ratio"}\NormalTok{,  }\StringTok{"solvency_ratio"}\NormalTok{, }\StringTok{"current_assets"}\NormalTok{,}
                          \StringTok{"fin_expenses"}\NormalTok{, }\StringTok{"net_income"}\NormalTok{, }\StringTok{"fin_cons"}\NormalTok{, }\StringTok{"fin_cons100"}\NormalTok{, }\StringTok{"inv"}\NormalTok{,}
                          \StringTok{"ICR_t"}\NormalTok{,  }\StringTok{"time"}\NormalTok{, }\StringTok{"real_SA"}\NormalTok{, }\StringTok{"shareholders_funds"}\NormalTok{, }\StringTok{"NEG_VA"}\NormalTok{,}
                          \StringTok{"ICR_failure"}\NormalTok{, }\StringTok{"profitability"}\NormalTok{, }\StringTok{"misallocated_fixed"}\NormalTok{, }
                          \StringTok{"interest_diff"}\NormalTok{)])}
\NormalTok{y <-}\StringTok{ }\KeywordTok{as.numeric}\NormalTok{(positive}\OperatorTok{$}\NormalTok{prediction}\OperatorTok{==}\DecValTok{1} \OperatorTok{&}\StringTok{ }\NormalTok{positive}\OperatorTok{$}\NormalTok{failure}\OperatorTok{==}\DecValTok{0}\NormalTok{)}

\NormalTok{mod <-}\StringTok{ }\KeywordTok{cv.glmnet}\NormalTok{(x , y, }\DataTypeTok{alpha=}\DecValTok{1}\NormalTok{)}
\KeywordTok{as.matrix}\NormalTok{(}\KeywordTok{coef}\NormalTok{(mod, mod}\OperatorTok{$}\NormalTok{lambda}\FloatTok{.1}\NormalTok{se))}
\end{Highlighting}
\end{Shaded}

\begin{verbatim}
##                                 1
## (Intercept)           0.493922108
## iso                  -0.013582090
## tfp_acf               0.012983613
## Number_of_patents     0.000000000
## Number_of_trademarks  0.000000000
## consdummy             0.087160526
## control               0.003484858
## nace_2                0.000000000
## fin_rev               0.000000000
## int_paid              0.000000000
## ebitda                0.000000000
## cash_flow             0.000000000
## depr                  0.000000000
## revenue               0.000000000
## total_assets          0.000000000
## long_term_debt        0.000000000
## employees             0.000000000
## added_value           0.000000000
## materials             0.000000000
## wage_bill             0.000000000
## loans                 0.000000000
## int_fixed_assets      0.000000000
## fixed_assets          0.000000000
## current_liabilities   0.000000000
## liquidity_ratio       0.000000000
## solvency_ratio        0.000000000
## current_assets        0.000000000
## fin_expenses          0.000000000
## net_income            0.000000000
## fin_cons              0.000000000
## fin_cons100           0.000000000
## inv                   0.000000000
## ICR_t                 0.000000000
## time                  0.000000000
## real_SA               0.000000000
## shareholders_funds    0.000000000
## NEG_VA                0.000000000
## ICR_failure           0.000000000
## profitability         0.000000000
## misallocated_fixed   -0.014444096
## interest_diff         0.000000000
\end{verbatim}

\begin{Shaded}
\begin{Highlighting}[]
\KeywordTok{row.names}\NormalTok{(}\KeywordTok{as.matrix}\NormalTok{(}\KeywordTok{coef}\NormalTok{(mod, mod}\OperatorTok{$}\NormalTok{lambda}\FloatTok{.1}\NormalTok{se)))[}\KeywordTok{which}\NormalTok{(}\KeywordTok{as.matrix}\NormalTok{(}\KeywordTok{coef}\NormalTok{(mod, mod}\OperatorTok{$}\NormalTok{lambda}\FloatTok{.1}\NormalTok{se))}\OperatorTok{!=}\DecValTok{0}\NormalTok{)]}
\end{Highlighting}
\end{Shaded}

\begin{verbatim}
## [1] "(Intercept)"        "iso"                "tfp_acf"           
## [4] "consdummy"          "control"            "misallocated_fixed"
\end{verbatim}

The indicator that seems to have the best disciminative power are the
productivity level, the consolidated variable and the zombie indicator
from Schivardi, Sette and Tabellini (2017)
(\textit{misallocated\_fixed}). Hence, ICR is not among these indicators
meaning that ICR alone can't catch the component of ``resistence'' among
highly distressed firms that makes them zombies.

\hypertarget{exclusion-of-icr-from-the-predictors}{%
\section{Exclusion of ICR from the
predictors}\label{exclusion-of-icr-from-the-predictors}}

A second central check, to understand how important is the contribute of
ICR to the predicted failure probability, is to evaluate the performance
of the model when we rule out its contribute as a predictor. This is
particularly interesting if we focus just on the subsample of
observations that have an ICR bigger than one (namely, are not defined
as zombie firms).

\par

In the following chunks we subset the data in a way that our favourite
ML algorithm (BART) is trained on samples where the number of units with
\(Y_i=1\) and \(Y_i=0\) is the same. This trick is used to improve the
predicive power of the model.

\begin{Shaded}
\begin{Highlighting}[]
\KeywordTok{attach}\NormalTok{(dati)}
\NormalTok{predictors <-}\StringTok{ }\NormalTok{dati[predictors]}
\NormalTok{output <-}\StringTok{ }\KeywordTok{as.numeric}\NormalTok{(failure }\OperatorTok{==}\StringTok{ }\DecValTok{1}\NormalTok{)}
\KeywordTok{detach}\NormalTok{(dati)}

\NormalTok{model1<-}\KeywordTok{data.frame}\NormalTok{(predictors, output)}
\KeywordTok{colnames}\NormalTok{(model1)[}\KeywordTok{length}\NormalTok{(model1)]<-}\StringTok{"output"}
\NormalTok{model1_}\DecValTok{0}\NormalTok{<-}\KeywordTok{subset}\NormalTok{(model1,model1}\OperatorTok{$}\NormalTok{output}\OperatorTok{==}\DecValTok{0}\NormalTok{) }
\NormalTok{model1_}\DecValTok{1}\NormalTok{<-}\KeywordTok{subset}\NormalTok{(model1,model1}\OperatorTok{$}\NormalTok{output}\OperatorTok{==}\DecValTok{1}\NormalTok{)}
\end{Highlighting}
\end{Shaded}

In this chunck we define a function for the BART algorithm.

\begin{Shaded}
\begin{Highlighting}[]
\NormalTok{### Define functions}
\NormalTok{bart <-}\StringTok{ }\ControlFlowTok{function}\NormalTok{(train1, test1, output)\{}
\NormalTok{  train1 <-}\StringTok{ }\KeywordTok{as.data.frame}\NormalTok{(train1)}
\NormalTok{  test1 <-}\StringTok{ }\KeywordTok{as.data.frame}\NormalTok{(test1)}
\NormalTok{  train1}\OperatorTok{$}\NormalTok{predictors <-}\StringTok{ }\NormalTok{train1[predictors]}
\NormalTok{  test1}\OperatorTok{$}\NormalTok{predictors <-}\StringTok{ }\NormalTok{test1[predictors]}
\NormalTok{  bart_machine<-}\KeywordTok{bartMachine}\NormalTok{(train1}\OperatorTok{$}\NormalTok{predictors,}\KeywordTok{as.factor}\NormalTok{(train1}\OperatorTok{$}\NormalTok{output)) }
\NormalTok{  fitted.results.bart <-}\StringTok{ }\DecValTok{1}\OperatorTok{-}\StringTok{ }\KeywordTok{round}\NormalTok{(}\KeywordTok{predict}\NormalTok{(bart_machine, test1}\OperatorTok{$}\NormalTok{predictors,}
                                          \DataTypeTok{type=}\StringTok{'prob'}\NormalTok{), }\DecValTok{6}\NormalTok{)}
  \CommentTok{#fitted.results.bart <-predict(bart_machine, test1$predictors,  type='class')}
\NormalTok{  prediction <-}\StringTok{ }\KeywordTok{as.numeric}\NormalTok{(fitted.results.bart }\OperatorTok{>}\StringTok{ }\FloatTok{0.5}\NormalTok{)}
\NormalTok{  cmsl=}\KeywordTok{table}\NormalTok{(test1}\OperatorTok{$}\NormalTok{output,prediction)}
\NormalTok{  res_all<-(cmsl[}\DecValTok{1}\NormalTok{,}\DecValTok{1}\NormalTok{]}\OperatorTok{+}\NormalTok{cmsl[}\DecValTok{2}\NormalTok{,}\DecValTok{2}\NormalTok{])}\OperatorTok{/}\NormalTok{(}\KeywordTok{length}\NormalTok{(test1}\OperatorTok{$}\NormalTok{output))}
\NormalTok{  res_}\DecValTok{1}\NormalTok{<-cmsl[}\DecValTok{2}\NormalTok{,}\DecValTok{2}\NormalTok{]}\OperatorTok{/}\NormalTok{(cmsl[}\DecValTok{2}\NormalTok{,}\DecValTok{1}\NormalTok{]}\OperatorTok{+}\NormalTok{cmsl[}\DecValTok{2}\NormalTok{,}\DecValTok{2}\NormalTok{])}
\NormalTok{  res_}\DecValTok{0}\NormalTok{<-cmsl[}\DecValTok{1}\NormalTok{,}\DecValTok{1}\NormalTok{]}\OperatorTok{/}\NormalTok{(cmsl[}\DecValTok{1}\NormalTok{,}\DecValTok{1}\NormalTok{]}\OperatorTok{+}\NormalTok{cmsl[}\DecValTok{1}\NormalTok{,}\DecValTok{2}\NormalTok{])}
\NormalTok{  res<-}\KeywordTok{cbind}\NormalTok{(res_all,res_}\DecValTok{1}\NormalTok{,res_}\DecValTok{0}\NormalTok{)}
\NormalTok{\}}
\end{Highlighting}
\end{Shaded}

And in this last chunck we fit the models (namely, one with all
predictors and one just on the subset of observations where ICR is more
than one {[}\textit{ICR\_failure}==0{]}) on two bootstraped samples from
the data.

\begin{Shaded}
\begin{Highlighting}[]
\NormalTok{### Vector of predictors}
\NormalTok{predictors <-}\StringTok{ }\KeywordTok{c}\NormalTok{(}\StringTok{"iso"}\NormalTok{, }\StringTok{"tfp_acf"}\NormalTok{,  }\StringTok{"Number_of_patents"}\NormalTok{, }\StringTok{"Number_of_trademarks"}\NormalTok{,}
                \StringTok{"consdummy"}\NormalTok{, }\StringTok{"control"}\NormalTok{, }\StringTok{"nace_2"}\NormalTok{,}\StringTok{"fin_rev"}\NormalTok{, }\StringTok{"int_paid"}\NormalTok{, }\StringTok{"ebitda"}\NormalTok{,}
                \StringTok{"cash_flow"}\NormalTok{, }\StringTok{"depr"}\NormalTok{, }\StringTok{"revenue"}\NormalTok{, }\StringTok{"total_assets"}\NormalTok{, }\StringTok{"long_term_debt"}\NormalTok{,}
                \StringTok{"employees"}\NormalTok{, }\StringTok{"added_value"}\NormalTok{, }\StringTok{"materials"}\NormalTok{, }\StringTok{"wage_bill"}\NormalTok{, }\StringTok{"loans"}\NormalTok{ ,}
                \StringTok{"int_fixed_assets"}\NormalTok{,  }\StringTok{"fixed_assets"}\NormalTok{, }\StringTok{"current_liabilities"}\NormalTok{,}
                \StringTok{"liquidity_ratio"}\NormalTok{,  }\StringTok{"solvency_ratio"}\NormalTok{, }\StringTok{"current_assets"}\NormalTok{,}
                 \StringTok{"fin_expenses"}\NormalTok{, }\StringTok{"net_income"}\NormalTok{, }\StringTok{"fin_cons"}\NormalTok{, }\StringTok{"fin_cons100"}\NormalTok{, }\StringTok{"inv"}\NormalTok{,}
                \StringTok{"ICR_t"}\NormalTok{, }\StringTok{"time"}\NormalTok{, }\StringTok{"real_SA"}\NormalTok{, }\StringTok{"shareholders_funds"}\NormalTok{, }\StringTok{"NEG_VA"}\NormalTok{,}
                \StringTok{"ICR_failure"}\NormalTok{, }\StringTok{"profitability"}\NormalTok{, }\StringTok{"misallocated_fixed"}\NormalTok{, }
                \StringTok{"interest_diff"}\NormalTok{)}

\NormalTok{### Create matrix to save bootstrapped results (N = B)}
\NormalTok{B=}\DecValTok{2}
\NormalTok{number_of_models=}\DecValTok{3}
\NormalTok{results<-}\KeywordTok{matrix}\NormalTok{(}\DataTypeTok{data=}\OtherTok{NA}\NormalTok{, }\DataTypeTok{nrow =}\NormalTok{ B, }\DataTypeTok{ncol =}\NormalTok{ number_of_models}\OperatorTok{*}\DecValTok{2}\NormalTok{)}

\KeywordTok{system.time}\NormalTok{(\{}
  \ControlFlowTok{for}\NormalTok{ (i }\ControlFlowTok{in}\NormalTok{ (}\DecValTok{1}\OperatorTok{:}\NormalTok{B)) \{}
\NormalTok{    ### Start loop}
    \KeywordTok{set.seed}\NormalTok{(}\DecValTok{123} \OperatorTok{+}\StringTok{ }\NormalTok{i) }\CommentTok{#setting seed for reproducible results}
    
    \CommentTok{# Repeatedly (B) draw subsamples}
\NormalTok{    model1_0sub <-}\StringTok{ }\NormalTok{model1_}\DecValTok{0}\NormalTok{[}\KeywordTok{sample}\NormalTok{(}\KeywordTok{seq_len}\NormalTok{(}\KeywordTok{nrow}\NormalTok{(model1_}\DecValTok{0}\NormalTok{)), }
                                   \DataTypeTok{size =} \KeywordTok{nrow}\NormalTok{(model1_}\DecValTok{1}\NormalTok{),}
                                   \DataTypeTok{replace =} \OtherTok{TRUE}\NormalTok{),]}
\NormalTok{    model1_full <-}\StringTok{ }\KeywordTok{rbind}\NormalTok{(model1_0sub,model1_}\DecValTok{1}\NormalTok{)}
    
    \CommentTok{# split into training and test (easier to handle)}
\NormalTok{    train_ind <-}\StringTok{ }\KeywordTok{sample}\NormalTok{(}\KeywordTok{seq_len}\NormalTok{(}\KeywordTok{nrow}\NormalTok{(model1_full)), }
                        \DataTypeTok{size =} \KeywordTok{nrow}\NormalTok{(model1_full)}\OperatorTok{*}\FloatTok{0.5}\NormalTok{) }
    
\NormalTok{    train1 <-}\StringTok{ }\NormalTok{model1_full[train_ind,]}
\NormalTok{    test1 <-}\StringTok{ }\NormalTok{model1_full[}\OperatorTok{-}\NormalTok{train_ind,]}
    
    \CommentTok{#BART ALL}
\NormalTok{    results[i,}\DecValTok{1}\OperatorTok{:}\DecValTok{3}\NormalTok{]<-}\KeywordTok{bart}\NormalTok{(train1,test1, output)}
    
    \CommentTok{#BART ICR = 0}
\NormalTok{    model1_full <-}\StringTok{ }\NormalTok{model1_full[}\KeywordTok{which}\NormalTok{(model1_full}\OperatorTok{$}\NormalTok{ICR_failure}\OperatorTok{==}\DecValTok{0}\NormalTok{),]}
\NormalTok{    train_ind <-}\StringTok{ }\KeywordTok{sample}\NormalTok{(}\KeywordTok{seq_len}\NormalTok{(}\KeywordTok{nrow}\NormalTok{(model1_full)), }
                        \DataTypeTok{size =} \KeywordTok{nrow}\NormalTok{(model1_full)}\OperatorTok{*}\FloatTok{0.5}\NormalTok{) }
    
\NormalTok{    train1 <-}\StringTok{ }\NormalTok{model1_full[train_ind,]}
\NormalTok{    test1 <-}\StringTok{ }\NormalTok{model1_full[}\OperatorTok{-}\NormalTok{train_ind,]}
    
\NormalTok{    results[i,}\DecValTok{4}\OperatorTok{:}\DecValTok{6}\NormalTok{]<-}\KeywordTok{bart}\NormalTok{(train1,test1, output)}
    
    \CommentTok{# LOADING STATUS}
    \KeywordTok{print}\NormalTok{(i}\OperatorTok{/}\NormalTok{B)}
\NormalTok{  \}}
\NormalTok{\})}
\end{Highlighting}
\end{Shaded}

As we can see from the outcome of this chunck the overall predictions
for the two models are not signficantly different at a significance
level of 0.95.

\begin{Shaded}
\begin{Highlighting}[]
\KeywordTok{summary}\NormalTok{(results)}
\end{Highlighting}
\end{Shaded}

\begin{verbatim}
##        V1               V2               V3               V4        
##  Min.   :0.8022   Min.   :0.8521   Min.   :0.7526   Min.   :0.7694  
##  1st Qu.:0.8044   1st Qu.:0.8522   1st Qu.:0.7569   1st Qu.:0.7731  
##  Median :0.8066   Median :0.8524   Median :0.7613   Median :0.7769  
##  Mean   :0.8066   Mean   :0.8524   Mean   :0.7613   Mean   :0.7769  
##  3rd Qu.:0.8088   3rd Qu.:0.8525   3rd Qu.:0.7656   3rd Qu.:0.7807  
##  Max.   :0.8111   Max.   :0.8527   Max.   :0.7699   Max.   :0.7844  
##        V5               V6        
##  Min.   :0.6988   Min.   :0.8124  
##  1st Qu.:0.7053   1st Qu.:0.8146  
##  Median :0.7118   Median :0.8167  
##  Mean   :0.7118   Mean   :0.8167  
##  3rd Qu.:0.7183   3rd Qu.:0.8188  
##  Max.   :0.7248   Max.   :0.8210
\end{verbatim}

\begin{Shaded}
\begin{Highlighting}[]
\KeywordTok{t.test}\NormalTok{(results[,}\DecValTok{1}\NormalTok{], results[,}\DecValTok{4}\NormalTok{])}
\end{Highlighting}
\end{Shaded}

\begin{verbatim}
## 
##  Welch Two Sample t-test
## 
## data:  results[, 1] and results[, 4]
## t = 3.4103, df = 1.6209, p-value = 0.1014
## alternative hypothesis: true difference in means is not equal to 0
## 95 percent confidence interval:
##  -0.01757694  0.07704034
## sample estimates:
## mean of x mean of y 
## 0.8066331 0.7769014
\end{verbatim}


\end{document}

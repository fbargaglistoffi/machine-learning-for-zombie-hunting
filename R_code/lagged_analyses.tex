\documentclass[]{article}
\usepackage{lmodern}
\usepackage{amssymb,amsmath}
\usepackage{ifxetex,ifluatex}
\usepackage{fixltx2e} % provides \textsubscript
\ifnum 0\ifxetex 1\fi\ifluatex 1\fi=0 % if pdftex
  \usepackage[T1]{fontenc}
  \usepackage[utf8]{inputenc}
\else % if luatex or xelatex
  \ifxetex
    \usepackage{mathspec}
  \else
    \usepackage{fontspec}
  \fi
  \defaultfontfeatures{Ligatures=TeX,Scale=MatchLowercase}
\fi
% use upquote if available, for straight quotes in verbatim environments
\IfFileExists{upquote.sty}{\usepackage{upquote}}{}
% use microtype if available
\IfFileExists{microtype.sty}{%
\usepackage{microtype}
\UseMicrotypeSet[protrusion]{basicmath} % disable protrusion for tt fonts
}{}
\usepackage[margin=1in]{geometry}
\usepackage{hyperref}
\hypersetup{unicode=true,
            pdftitle={Machine Learning Analysis with Lagged Predictors},
            pdfauthor={Falco J. Bargagli-Stoffi, Massimo Riccaboni, Armando Rungi},
            pdfborder={0 0 0},
            breaklinks=true}
\urlstyle{same}  % don't use monospace font for urls
\usepackage{color}
\usepackage{fancyvrb}
\newcommand{\VerbBar}{|}
\newcommand{\VERB}{\Verb[commandchars=\\\{\}]}
\DefineVerbatimEnvironment{Highlighting}{Verbatim}{commandchars=\\\{\}}
% Add ',fontsize=\small' for more characters per line
\usepackage{framed}
\definecolor{shadecolor}{RGB}{248,248,248}
\newenvironment{Shaded}{\begin{snugshade}}{\end{snugshade}}
\newcommand{\AlertTok}[1]{\textcolor[rgb]{0.94,0.16,0.16}{#1}}
\newcommand{\AnnotationTok}[1]{\textcolor[rgb]{0.56,0.35,0.01}{\textbf{\textit{#1}}}}
\newcommand{\AttributeTok}[1]{\textcolor[rgb]{0.77,0.63,0.00}{#1}}
\newcommand{\BaseNTok}[1]{\textcolor[rgb]{0.00,0.00,0.81}{#1}}
\newcommand{\BuiltInTok}[1]{#1}
\newcommand{\CharTok}[1]{\textcolor[rgb]{0.31,0.60,0.02}{#1}}
\newcommand{\CommentTok}[1]{\textcolor[rgb]{0.56,0.35,0.01}{\textit{#1}}}
\newcommand{\CommentVarTok}[1]{\textcolor[rgb]{0.56,0.35,0.01}{\textbf{\textit{#1}}}}
\newcommand{\ConstantTok}[1]{\textcolor[rgb]{0.00,0.00,0.00}{#1}}
\newcommand{\ControlFlowTok}[1]{\textcolor[rgb]{0.13,0.29,0.53}{\textbf{#1}}}
\newcommand{\DataTypeTok}[1]{\textcolor[rgb]{0.13,0.29,0.53}{#1}}
\newcommand{\DecValTok}[1]{\textcolor[rgb]{0.00,0.00,0.81}{#1}}
\newcommand{\DocumentationTok}[1]{\textcolor[rgb]{0.56,0.35,0.01}{\textbf{\textit{#1}}}}
\newcommand{\ErrorTok}[1]{\textcolor[rgb]{0.64,0.00,0.00}{\textbf{#1}}}
\newcommand{\ExtensionTok}[1]{#1}
\newcommand{\FloatTok}[1]{\textcolor[rgb]{0.00,0.00,0.81}{#1}}
\newcommand{\FunctionTok}[1]{\textcolor[rgb]{0.00,0.00,0.00}{#1}}
\newcommand{\ImportTok}[1]{#1}
\newcommand{\InformationTok}[1]{\textcolor[rgb]{0.56,0.35,0.01}{\textbf{\textit{#1}}}}
\newcommand{\KeywordTok}[1]{\textcolor[rgb]{0.13,0.29,0.53}{\textbf{#1}}}
\newcommand{\NormalTok}[1]{#1}
\newcommand{\OperatorTok}[1]{\textcolor[rgb]{0.81,0.36,0.00}{\textbf{#1}}}
\newcommand{\OtherTok}[1]{\textcolor[rgb]{0.56,0.35,0.01}{#1}}
\newcommand{\PreprocessorTok}[1]{\textcolor[rgb]{0.56,0.35,0.01}{\textit{#1}}}
\newcommand{\RegionMarkerTok}[1]{#1}
\newcommand{\SpecialCharTok}[1]{\textcolor[rgb]{0.00,0.00,0.00}{#1}}
\newcommand{\SpecialStringTok}[1]{\textcolor[rgb]{0.31,0.60,0.02}{#1}}
\newcommand{\StringTok}[1]{\textcolor[rgb]{0.31,0.60,0.02}{#1}}
\newcommand{\VariableTok}[1]{\textcolor[rgb]{0.00,0.00,0.00}{#1}}
\newcommand{\VerbatimStringTok}[1]{\textcolor[rgb]{0.31,0.60,0.02}{#1}}
\newcommand{\WarningTok}[1]{\textcolor[rgb]{0.56,0.35,0.01}{\textbf{\textit{#1}}}}
\usepackage{graphicx,grffile}
\makeatletter
\def\maxwidth{\ifdim\Gin@nat@width>\linewidth\linewidth\else\Gin@nat@width\fi}
\def\maxheight{\ifdim\Gin@nat@height>\textheight\textheight\else\Gin@nat@height\fi}
\makeatother
% Scale images if necessary, so that they will not overflow the page
% margins by default, and it is still possible to overwrite the defaults
% using explicit options in \includegraphics[width, height, ...]{}
\setkeys{Gin}{width=\maxwidth,height=\maxheight,keepaspectratio}
\IfFileExists{parskip.sty}{%
\usepackage{parskip}
}{% else
\setlength{\parindent}{0pt}
\setlength{\parskip}{6pt plus 2pt minus 1pt}
}
\setlength{\emergencystretch}{3em}  % prevent overfull lines
\providecommand{\tightlist}{%
  \setlength{\itemsep}{0pt}\setlength{\parskip}{0pt}}
\setcounter{secnumdepth}{0}
% Redefines (sub)paragraphs to behave more like sections
\ifx\paragraph\undefined\else
\let\oldparagraph\paragraph
\renewcommand{\paragraph}[1]{\oldparagraph{#1}\mbox{}}
\fi
\ifx\subparagraph\undefined\else
\let\oldsubparagraph\subparagraph
\renewcommand{\subparagraph}[1]{\oldsubparagraph{#1}\mbox{}}
\fi

%%% Use protect on footnotes to avoid problems with footnotes in titles
\let\rmarkdownfootnote\footnote%
\def\footnote{\protect\rmarkdownfootnote}

%%% Change title format to be more compact
\usepackage{titling}

% Create subtitle command for use in maketitle
\providecommand{\subtitle}[1]{
  \posttitle{
    \begin{center}\large#1\end{center}
    }
}

\setlength{\droptitle}{-2em}

  \title{Machine Learning Analysis with Lagged Predictors}
    \pretitle{\vspace{\droptitle}\centering\huge}
  \posttitle{\par}
    \author{Falco J. Bargagli-Stoffi, Massimo Riccaboni, Armando Rungi}
    \preauthor{\centering\large\emph}
  \postauthor{\par}
      \predate{\centering\large\emph}
  \postdate{\par}
    \date{23/2/2020}


\begin{document}
\maketitle

\hypertarget{introduction}{%
\section{Introduction}\label{introduction}}

This \texttt{R Markdown} file reproduces the lagged machine learning
analysis for the paper
\textit{"Machine learning for zombie hunting. Firms' failures, financial constraints, and misallocation"}
by Falco J. Bargagli-Stoffi (IMT School for Advanced Studies/KU Leuven),
Massimo Riccaboni (IMT School for Advanced Studies) and Armando Rungi
(IMT School for Advanced Studies).

\hypertarget{r-markdown}{%
\subsection{R Markdown}\label{r-markdown}}

This is an \texttt{R Markdown} document. Markdown is a simple formatting
syntax for authoring HTML, PDF, and MS Word documents. For more details
on using \texttt{R Markdown} see \url{http://rmarkdown.rstudio.com}.

When you click the \textbf{Knit} button a document will be generated
that includes both content as well as the output of any embedded
\texttt{R} code chunks within the document. You can embed an R code
chunks like the following.

\hypertarget{packages-upload}{%
\subsection{Packages Upload}\label{packages-upload}}

The following packages and functions are the ones used for the analyses
performed in the \texttt{R} code. The \texttt{functions.R} file contains
the functions \texttt{F1\_score}, \texttt{balanced\_accuracy},
\texttt{model\_compare} and \texttt{DtD} that were developed to
reproduce the following analyses.

\begin{Shaded}
\begin{Highlighting}[]
\KeywordTok{rm}\NormalTok{(}\DataTypeTok{list=}\KeywordTok{ls}\NormalTok{()) }\CommentTok{# to clean the memeory}
\KeywordTok{memory.limit}\NormalTok{(}\DataTypeTok{size=}\DecValTok{1000000}\NormalTok{)}
\end{Highlighting}
\end{Shaded}

\begin{verbatim}
## [1] 1e+06
\end{verbatim}

\begin{Shaded}
\begin{Highlighting}[]
\KeywordTok{options}\NormalTok{(}\DataTypeTok{java.parameters =} \StringTok{"-Xmx15000m"}\NormalTok{)}
\KeywordTok{library}\NormalTok{(rJava)}
\KeywordTok{library}\NormalTok{(bartMachine)}
\KeywordTok{library}\NormalTok{(haven)}
\KeywordTok{library}\NormalTok{(plyr)}
\KeywordTok{library}\NormalTok{(dplyr)}
\KeywordTok{library}\NormalTok{(PRROC)}
\KeywordTok{library}\NormalTok{(rpart)}
\KeywordTok{library}\NormalTok{(party)}
\KeywordTok{library}\NormalTok{(caret)}
\KeywordTok{library}\NormalTok{(devtools)}
\KeywordTok{library}\NormalTok{(SuperLearner)}
\KeywordTok{library}\NormalTok{(Metrics)}
\KeywordTok{library}\NormalTok{(pROC)}
\KeywordTok{library}\NormalTok{(Hmisc)}
\KeywordTok{source}\NormalTok{(}\StringTok{'functions.R'}\NormalTok{)}
\end{Highlighting}
\end{Shaded}

\hypertarget{bart-mia}{%
\subsection{BART-mia}\label{bart-mia}}

Run a Bayesian Additive Regression Tree analysis by using the overall
data sample (no need to omit the observations with missing values).

\begin{Shaded}
\begin{Highlighting}[]
\KeywordTok{set.seed}\NormalTok{(}\DecValTok{2020}\NormalTok{)}
\NormalTok{sample <-}\StringTok{ }\KeywordTok{sample}\NormalTok{(}\KeywordTok{seq_len}\NormalTok{(}\KeywordTok{nrow}\NormalTok{(data_italy)),}
                 \DataTypeTok{size =} \KeywordTok{nrow}\NormalTok{(omitted),}
                 \DataTypeTok{replace=}\OtherTok{FALSE}\NormalTok{) }
\NormalTok{data_italy_bart <-}\StringTok{ }\NormalTok{data_italy[sample,]}
\end{Highlighting}
\end{Shaded}

Select the same number of observations as in the previous models for the
training and testing samples.

\begin{Shaded}
\begin{Highlighting}[]
\KeywordTok{set.seed}\NormalTok{(}\DecValTok{2020}\NormalTok{)}
\NormalTok{train_sample <-}\StringTok{ }\KeywordTok{sample}\NormalTok{(}\KeywordTok{seq_len}\NormalTok{(}\KeywordTok{nrow}\NormalTok{(data_italy_bart)),}
                       \DataTypeTok{size =} \KeywordTok{nrow}\NormalTok{(data_italy_bart )}\OperatorTok{*}\FloatTok{0.9}\NormalTok{, }\DataTypeTok{replace=}\OtherTok{FALSE}\NormalTok{) }
\NormalTok{train_bart <-}\StringTok{ }\NormalTok{data_italy_bart[train_sample,]}
\NormalTok{test_bart <-}\StringTok{ }\NormalTok{data_italy_bart[}\OperatorTok{-}\NormalTok{train_sample,]}
\NormalTok{train_bart}\OperatorTok{$}\NormalTok{X <-}\StringTok{ }\KeywordTok{as.data.frame}\NormalTok{(train_bart[predictors])}
\NormalTok{test_bart}\OperatorTok{$}\NormalTok{X <-}\StringTok{ }\KeywordTok{as.data.frame}\NormalTok{(test_bart[predictors])}
\end{Highlighting}
\end{Shaded}

Run the analysis.

\begin{Shaded}
\begin{Highlighting}[]
\KeywordTok{system.time}\NormalTok{(\{}
\NormalTok{bart_machine<-}\KeywordTok{bartMachine}\NormalTok{(train_bart}\OperatorTok{$}\NormalTok{X,}
                          \KeywordTok{as.factor}\NormalTok{(train_bart}\OperatorTok{$}\NormalTok{failure),}
                          \DataTypeTok{use_missing_data=}\OtherTok{TRUE}\NormalTok{) }
\NormalTok{\})}
\end{Highlighting}
\end{Shaded}

Depict the performance measures by running the following chunks.

\begin{Shaded}
\begin{Highlighting}[]
\NormalTok{fitted.results.bart <-}\StringTok{ }\DecValTok{1}\OperatorTok{-}\StringTok{ }\KeywordTok{round}\NormalTok{(}\KeywordTok{predict}\NormalTok{(bart_machine,}
\NormalTok{                                        test_bart}\OperatorTok{$}\NormalTok{X,}
                                        \DataTypeTok{type=}\StringTok{'prob'}\NormalTok{), }\DecValTok{6}\NormalTok{)}
\end{Highlighting}
\end{Shaded}

\begin{Shaded}
\begin{Highlighting}[]
\CommentTok{#Roc}
\NormalTok{fg.bart<-fitted.results.bart[test_bart}\OperatorTok{$}\NormalTok{failure}\OperatorTok{==}\DecValTok{1}\NormalTok{] }
\NormalTok{bg.bart<-fitted.results.bart[test_bart}\OperatorTok{$}\NormalTok{failure}\OperatorTok{==}\DecValTok{0}\NormalTok{]}
\end{Highlighting}
\end{Shaded}

\begin{figure}
\centering
\includegraphics{G:/Il mio Drive/Research/Italian Firms/Zombie Hunting New Data/roc_bart.jpg}
\caption{Area under the ROC curve, BART}
\end{figure}

\begin{Shaded}
\begin{Highlighting}[]
\NormalTok{roc_bart<-}\KeywordTok{roc.curve}\NormalTok{(}\DataTypeTok{scores.class0 =}\NormalTok{ fg.bart,}
                    \DataTypeTok{scores.class1 =}\NormalTok{ bg.bart,}
                    \DataTypeTok{curve =}\NormalTok{ T)}
\KeywordTok{plot}\NormalTok{(roc_bart)}
\end{Highlighting}
\end{Shaded}

\begin{figure}
\centering
\includegraphics{G:/Il mio Drive/Research/Italian Firms/Zombie Hunting New Data/pr_bart.jpg}
\caption{Area under the PR curve, BART}
\end{figure}

\begin{Shaded}
\begin{Highlighting}[]
\NormalTok{pr_bart<-}\KeywordTok{pr.curve}\NormalTok{(}\DataTypeTok{scores.class0 =}\NormalTok{ fg.bart,}
                  \DataTypeTok{scores.class1 =}\NormalTok{ bg.bart,}
                  \DataTypeTok{curve =}\NormalTok{ T)}
\KeywordTok{plot}\NormalTok{(pr_bart)}
\end{Highlighting}
\end{Shaded}

\begin{Shaded}
\begin{Highlighting}[]
\CommentTok{#Get Accurancy}
\NormalTok{fitted.bart <-}\StringTok{ }\KeywordTok{ifelse}\NormalTok{(fitted.results.bart}\OperatorTok{>}\StringTok{ }\FloatTok{0.5}\NormalTok{, }\DecValTok{1}\NormalTok{, }\DecValTok{0}\NormalTok{)}
\NormalTok{f1_bart <-}\StringTok{ }\KeywordTok{f1_score}\NormalTok{(fitted.bart,}
\NormalTok{                    test_bart}\OperatorTok{$}\NormalTok{failure,}
                    \DataTypeTok{positive.class=}\StringTok{"1"}\NormalTok{)}
\end{Highlighting}
\end{Shaded}

\begin{Shaded}
\begin{Highlighting}[]
\NormalTok{balanced_accuracy_bart <-}\StringTok{ }\KeywordTok{balanced_accuracy}\NormalTok{(}\KeywordTok{as.matrix}\NormalTok{(}\KeywordTok{table}\NormalTok{(fitted.bart,}
\NormalTok{                                                    test_bart}\OperatorTok{$}\NormalTok{failure)))}
\NormalTok{accuracy_bart <-}\StringTok{ }\KeywordTok{as.data.frame}\NormalTok{(}\KeywordTok{rbind}\NormalTok{(}\KeywordTok{postResample}\NormalTok{(}\KeywordTok{as.double}\NormalTok{(fitted.bart),}
\NormalTok{                                                    test_bart}\OperatorTok{$}\NormalTok{failure)))}
\end{Highlighting}
\end{Shaded}

\begin{Shaded}
\begin{Highlighting}[]
\NormalTok{bart_fit <-}\StringTok{ }\KeywordTok{as.data.frame}\NormalTok{(}\KeywordTok{cbind}\NormalTok{(roc_bart}\OperatorTok{$}\NormalTok{auc,}
\NormalTok{                                pr_bart}\OperatorTok{$}\NormalTok{auc.integral,}
\NormalTok{                                f1_bart,}
\NormalTok{                                balanced_accuracy_bart,}
\NormalTok{                                accuracy_bart}\OperatorTok{$}\NormalTok{Rsquared))}
\KeywordTok{colnames}\NormalTok{(bart_fit) <-}\StringTok{ }\KeywordTok{c}\NormalTok{(}\StringTok{"AUC"}\NormalTok{, }\StringTok{"PR"}\NormalTok{, }\StringTok{"f1-score"}\NormalTok{, }\StringTok{"BACC"}\NormalTok{, }\StringTok{"Rsquared"}\NormalTok{)}
\end{Highlighting}
\end{Shaded}

\hypertarget{save-results}{%
\subsection{Save Results}\label{save-results}}

\begin{Shaded}
\begin{Highlighting}[]
\NormalTok{model_results <-}\StringTok{ }\KeywordTok{rbind}\NormalTok{(logit_fit, ctree_fit, rf_fit, sl_fit, bart_fit)}
\KeywordTok{write.csv}\NormalTok{(model_results, }\DataTypeTok{file =} \StringTok{"model_results.csv"}\NormalTok{)}
\end{Highlighting}
\end{Shaded}

\hypertarget{model-comparison}{%
\section{Model Comparison}\label{model-comparison}}

Here, we run an empirical horse race where we define two competitors
(benchmark or ``usual methods'') of our preferred BART methodology.
Natural candidates are ``default probability predictors''
(credit-ratings type of measures) such as:

\begin{enumerate}
\item  Altman Z-score;
\item  Distance-to-Default.
\end{enumerate}

As these measures do not provide direct predictions for failed firms, we
create a series of dummy variables in the following way:
\begin{equation}
        Z-dummy_i=\begin{cases}
        1 &  \text{if $Z-score_i \leq q$}, \nonumber \\
        0 &   \text{otherwise}
        \end{cases}
\end{equation} and \begin{equation}
        DtD-dummy_i=\begin{cases}
        1 &  \text{if $DtD_i \leq q$}, \nonumber \\
        0 &   \text{otherwise}
        \end{cases}
\end{equation} where \(q\) is the 1st to 10th percentile distribution of
the Z-score and the DtD measures, respectively.

By doing so, we assume to be predicted as ``failed'', those observations
with values on the left tails of the Z and DtD measures.

We create these variables on the testing set and then we compare their
performance, in terms of precision and false discovery rate (FDR), with
the one of BART.

\hypertarget{z-score}{%
\subsection{Z-score}\label{z-score}}

\begin{Shaded}
\begin{Highlighting}[]
\NormalTok{precision_bart <-}\StringTok{ }\KeywordTok{as.data.frame}\NormalTok{(}\KeywordTok{t}\NormalTok{(}\KeywordTok{model_compare}\NormalTok{(fitted.bart,}
\NormalTok{                                  test_bart}\OperatorTok{$}\NormalTok{failure, }\StringTok{"1"}\NormalTok{)))}
\KeywordTok{colnames}\NormalTok{(precision_bart) <-}\StringTok{ }\KeywordTok{c}\NormalTok{(}\StringTok{"Precision BART"}\NormalTok{, }\StringTok{"FDR BART"}\NormalTok{)}
\KeywordTok{write.csv}\NormalTok{(precision_bart, }\StringTok{"precision_and_fdr_bart.csv"}\NormalTok{)}
\end{Highlighting}
\end{Shaded}

\begin{Shaded}
\begin{Highlighting}[]
\NormalTok{seq <-}\StringTok{ }\KeywordTok{seq}\NormalTok{(}\DecValTok{1}\NormalTok{, }\DecValTok{10}\NormalTok{, }\DecValTok{1}\NormalTok{)}
\NormalTok{precision_zscore <-}\StringTok{ }\KeywordTok{matrix}\NormalTok{(}\OtherTok{NA}\NormalTok{, }\DataTypeTok{ncol =} \DecValTok{2}\NormalTok{, }\DataTypeTok{nrow =} \KeywordTok{length}\NormalTok{(seq))}
\ControlFlowTok{for}\NormalTok{(i }\ControlFlowTok{in}\NormalTok{ seq) \{}
\NormalTok{  failed_zscore <-}\StringTok{ }\KeywordTok{ifelse}\NormalTok{(test}\OperatorTok{$}\NormalTok{Z_score }\OperatorTok{<=}\StringTok{ }\KeywordTok{quantile}\NormalTok{(test}\OperatorTok{$}\NormalTok{Z_score, i}\OperatorTok{/}\DecValTok{100}\NormalTok{),}
                                                                  \DecValTok{1}\NormalTok{, }\DecValTok{0}\NormalTok{)}
\NormalTok{  precision_zscore[i, }\KeywordTok{c}\NormalTok{(}\DecValTok{1}\OperatorTok{:}\DecValTok{2}\NormalTok{)] <-}\StringTok{ }\KeywordTok{t}\NormalTok{(}\KeywordTok{model_compare}\NormalTok{(failed_zscore,}
\NormalTok{                                                 test}\OperatorTok{$}\NormalTok{failure, }\StringTok{"1"}\NormalTok{))}
\NormalTok{\}}
\end{Highlighting}
\end{Shaded}

\hypertarget{mertons-distance-to-default-dtd}{%
\subsection{Merton's Distance-to-Default
(DtD)}\label{mertons-distance-to-default-dtd}}

The average equity volatility in the considered time series is 27.9120
(Bank of Italy), while for the average risk free interest rate we use
the long term government bond yields" EMU (Eurostat) that has a value of
4.2937.

\begin{Shaded}
\begin{Highlighting}[]
\NormalTok{dtd_score <-}\StringTok{ }\KeywordTok{DtD}\NormalTok{(}\DataTypeTok{mcap =}\NormalTok{ test}\OperatorTok{$}\NormalTok{shareholders_funds, }\DataTypeTok{debt =}\NormalTok{ test}\OperatorTok{$}\NormalTok{long_term_debt, }\DataTypeTok{vol =} \FloatTok{27.9120}\NormalTok{, }\DataTypeTok{r =} \FloatTok{4.2937}\NormalTok{)}
\end{Highlighting}
\end{Shaded}

\begin{Shaded}
\begin{Highlighting}[]
\NormalTok{seq <-}\StringTok{ }\KeywordTok{seq}\NormalTok{(}\DecValTok{1}\NormalTok{, }\DecValTok{10}\NormalTok{, }\DecValTok{1}\NormalTok{)}
\NormalTok{precision_dtd_score <-}\StringTok{ }\KeywordTok{matrix}\NormalTok{(}\OtherTok{NA}\NormalTok{, }\DataTypeTok{ncol =} \DecValTok{2}\NormalTok{, }\DataTypeTok{nrow =} \KeywordTok{length}\NormalTok{(seq))}
\ControlFlowTok{for}\NormalTok{(i }\ControlFlowTok{in}\NormalTok{ seq) \{}
\NormalTok{  failed_dtd_score <-}\StringTok{ }\KeywordTok{ifelse}\NormalTok{(dtd_score }\OperatorTok{<=}\StringTok{ }\KeywordTok{quantile}\NormalTok{(dtd_score, i}\OperatorTok{/}\DecValTok{100}\NormalTok{), }\DecValTok{1}\NormalTok{, }\DecValTok{0}\NormalTok{)}
\NormalTok{  precision_dtd_score[i, }\KeywordTok{c}\NormalTok{(}\DecValTok{1}\OperatorTok{:}\DecValTok{2}\NormalTok{)] <-}\StringTok{ }\KeywordTok{t}\NormalTok{(}\KeywordTok{model_compare}\NormalTok{(failed_dtd_score,}
\NormalTok{                                                    test}\OperatorTok{$}\NormalTok{failure, }\StringTok{"1"}\NormalTok{))}
\NormalTok{\}}
\end{Highlighting}
\end{Shaded}

\begin{Shaded}
\begin{Highlighting}[]
\NormalTok{precision_models <-}\StringTok{ }\KeywordTok{as.data.frame}\NormalTok{(}\KeywordTok{cbind}\NormalTok{(precision_dtd_score, precision_zscore))}
\KeywordTok{colnames}\NormalTok{(precision_models) <-}\StringTok{ }\KeywordTok{c}\NormalTok{(}\StringTok{"Precision DtD"}\NormalTok{,}
                                \StringTok{"FDR DtD"}\NormalTok{,}
                                \StringTok{"Precision Z-score"}\NormalTok{,}
                                \StringTok{"FDR Z-score"}\NormalTok{)}
\KeywordTok{write.csv}\NormalTok{(precision_models, }\StringTok{"precision_and_fdr_scores.csv"}\NormalTok{)}
\end{Highlighting}
\end{Shaded}


\end{document}
